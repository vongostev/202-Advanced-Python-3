\documentclass[a4paper, 10pt]{article}
\usepackage[T2A]{fontenc}
\usepackage[utf8]{inputenc}
\usepackage[russian,english]{babel}
\usepackage{amsfonts}
\usepackage{amssymb}
\usepackage{amsmath}
\parindent=0cm
\begin{document}
	
\section{Документация}
	\textbf{Исполняемый файл:}\\
	<<Program.py>>;\\
	\textbf{Файл вывода (по умолчанию):}\\ 
	<<Выкладки.tex>>,\\
	файл сбрасывается при каждом исполнении алгоритма,\\
	дополнительная модификация фала для успешного открытя в редакторе и последующей трансляции в PDF не требуется;\\
	\textbf{Строгие зависимости:}\\
	sympy (основа символьных вычислений),\\ 
	einsteinpy.symbolic (встроенные расчёты базовых геометрических объектов),\\ 
	time (измерение времени выполнения программы);\\
	\textbf{Особенности:}\\
	1) В целях повышения гибкости и облегчения восприятия, настройка параметров производится в обособленном символами <<!>> участке кода с заглавием \linebreak <<ОСНОВНЫЕ~ПАРАМЕТРЫ:>>.\\
	2) Ключи вида <<key\_calc\_\dots>> отвечают за вычисления, <<key\_write\_\dots>> -- за выводы. Вывод без вычисления запрещён.\\
	3) Класс <<Output\_Object>> является служебным и предназначин исключительно для упрощения кода.\\
	4) Геометрические расчёты принимают в качестве данных ковариантную метрику. В функционал входят вычисление и вывод в файл контравариантной метрики (обратная матрица), символов Кристоффеля II рода, тензора кривизны Римана $R_{abcd}$, его скалярного квадрата, тензора Риччи $R_{ab}$, его скалярного квадрата, скалярной кривизны и тензора Эйнштейна $G_{ab}$.
\section{Справочные материалы}
	!!!Везде далее используется соглашение Эйнштейна о повторяющихся индексах.!!!
	\subsection{Определения}
	Ковариантная метрика: $g_{\mu\nu}$.\\
	Контравариантная метрика: $g^{\mu\nu}$, $g_{ca}g^{ab} = \delta^b_c$.\\
	Символы Кристоффеля II рода: $\Gamma^i_{kl} = \frac{1}{2}g^{im}\left( \partial_l g_{mk} + \partial_k g_{ml} - \partial_m g_{kl} \right)$.\\
	Тензор кривизны Римана : 
	$$R_{\mu k \nu m} = g_{\mu\rho} {R^\rho}_{k \nu m}, \;\; {R^\rho}_{\sigma\mu\nu} = \partial_{\mu}\Gamma^\rho_{\nu\sigma} - \partial_{\nu}\Gamma^\rho_{\mu\sigma} + \Gamma^\rho_{\mu\lambda}\Gamma^{\lambda}_{\nu\sigma} - \Gamma^\rho_{\nu\lambda}\Gamma^\lambda_{\mu\sigma}.$$
	Его скалярный квадрат: $R_{abcd}R^{abcd}$.\\
	Тензор Риччи: $R_{kl} = {R^\rho}_{k \rho l}$.\\
	Его скалярный квадрат: $R_{ab}R^{ab}$.\\
	Скалярная кривизна: $R = R_{km}g^{km}$.\\
	Тензор Эйнштейна: $G_{ab} = R_{ab} - \frac{1}{2}Rg_{ab}$.
	\subsection{Некоторые замечания}
	Далее, где нужно, $\operatorname{dim} = 4$.\\
	№0\\
	Всякий симметричный тензор ранга 2 имеет здесь 10 алгебраически независимых компонент.\\
	№1\\
	Метрический тензор симметричен и свёртка с ним производит поднятие-опускание индексов.\\
	Почти всюду имеет место $\det g_{\mu\nu}= 0$.\\
	№2\\
	!!!Символы Кристоффеля тензорами не являются!!!\\
	$\Gamma^a_{bc} = \Gamma^a_{cb}$\\
	№3\\
	Cимметрии тензора Римана: $R_{abcd}=-R_{bacd}=-R_{abdc}=R_{cdab}$.\\
	Первое тождество Бианки: $R_{abcd}+R_{acdb}+R_{adbc}=0$.\\
	Суммарно 20 алгебраически независимых компонент из общего их числа 256: 235 зависмы из-за симметрий и ещё 1 из-за тождества Бианки.\\
	№4
	Скалярный квадрат любого тензора -- скаляр. Скалярная кривизна -- скаляр.\\
	№5\\
	$R_{ab} = R_{ba}$, 	$G_{ab} = G_{ba}$.\\
	$\nabla_a G^{ab} \equiv 0$. Тензор Эйнштейна представляет собой бездивергентную часть тензора Риччи.

\end{document}