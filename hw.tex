\documentclass{article}
\usepackage{amsmath}
\usepackage{amssymb}
\def\stackbelow#1#2{\underset{\displaystyle\overset{\displaystyle\shortparallel}{#2}}{#1}}
\usepackage[T2A]{fontenc}    
\usepackage[utf8]{inputenc} 
\usepackage{graphicx}
\usepackage[colorlinks=true, allcolors=blue]{hyperref}
\usepackage{float}

\usepackage[english,russian]{babel}
\title{Домашнее задание по программированию}
\author{Галина Прилипко, 204 гр.}
\begin{document}
\maketitle
\begin{abstract}
  Это вводный абзац в начале документа.
\end{abstract}

\section{10 формул}

Здесь может быть текст, в котором я могу цитировать других авторов. Например, я хочу упоминуть статью про черную дыру в центре М87 \cite{M87}. Или если например цитата из статьи про золотых рыбок \cite{goldfish}.
Или статья про импорт древесины в северную Ирландию \cite{ancient}.

формула 1: 
$\ V_i(t+\delta t) = V_i(t)+\delta t J(t) \frac{S^i_surface}{\sum_{j=1}^{n} S^j_surface}$

формула 2: 
$\frac{d V_1}{d t} = J(t) \frac{V_1^{2/3}}{\sum_{i=1}^{n} V_1^{2/3}}$

формула 3: 
$\frac{dV_i}{dt}= J(t) \frac{4 \pi r_n^2}{4 \pi \sum_{j=1}^{n}{r_j^2}}$

формула 4: 
$\frac{d(\frac{4\pi r_i^3}{3})}{dt}=\ J(t)\frac{r_i^2}{\sum_{j=1}^{n}r_j^2}$

формула 5: 
$\frac{dr_i}{dt}=\ J(t)\frac{1}{4\pi\sum_{j=1}^{n}r_j^2}$

формула 6: 
$\ V_i(t+\delta t) = V_i(t)+\delta t J(t) \frac{S^i_surface}{\sum_{j=1}^{n} S^j_surface}$

у меня закончились формулы из курсовой, поэтому далее пойдут просто какие-то формулы
формула 7: 
$\ E_y = \frac{\hbar^2}{2 I} \stackbelow{J}{1} (J+1)$

формула 8: 
\begin{equation*}
 \begin{cases}
   \frac{\partial u}{\partial t} + x^2 \frac{\partial^2 u}{\partial x^2} = a, 
   \\
   u\Biggr|_{1} = 1 - x^2.
 \end{cases}
\end{equation*}

формула 9: 
$ grad(u) = \nabla u = \vec{i} \frac{\partial u}{\partial x} + \vec{j} \frac{\partial u}{\partial y}$

формула 10: 
$\begin{pmatrix}
  D_x\\
  D_y\\
  D_z
\end{pmatrix}
= \epsilon_0
\begin{pmatrix}
  \ \epsilon_x & 0 & 0\\
  \ 0 & \epsilon_y & 0\\
  \ 0& 0 & \epsilon_z
\end{pmatrix}
\begin{pmatrix}
  E_x\\
  E_y\\
  E_z
\end{pmatrix}$



\section{Рисунок}
Здесь показан график изменения обьемов дроплетов 

\begin{figure}[H]
\centering
\includegraphics[width=0.5\textwidth]{fig1.png}
\caption{\label{fig:fig1}Это график из моей курсовой}
\end{figure}

\section{3 библиографические ссылки}
\bibliographystyle{alpha}
\bibliography{bib}
\end{document}
