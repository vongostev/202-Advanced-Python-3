\documentclass{article}
\usepackage[T2A]{fontenc}
\usepackage[english, russian]{babel}
\usepackage[utf8]{inputenc}
\usepackage{amsmath}
\usepackage{graphicx}

\title{Рассуждения об интегралах}

\begin{document}
\maketitle

Рассмотрим интеграл
 $$\int_1^2\frac{xdx}{1-x^2}=\frac{1}{2}\int_1^2\frac{d(x^2)}{1-x^2}=-\frac{1}{2}\int_1^2\frac{d(1-x^2)}{1-x^2}=-\frac{1}{2}\ln|1-x^2|\Big|_1^2=$$
 $$=-\frac12\ln3+\lim\limits_{x\to1}\frac12\ln|1-x^2|=-\infty,$$
данный интеграл является несобственным интегралом второго рода, в нем подынтегральная функция не ограничена в нижнем пределе интегрирования, и он расходится.

Следующий интеграл вычислим  интегрированием по частям:
$$\int_0^{e-1}\ln(x+1)dx=x\cdot\ln(x+1)\Big|_0^{e-1}-\int_0^{e-1}xd(\ln(x+1))=$$
$$=(e-1)\ln e-0\cdot\ln 1-\int_0^{e-1}\frac{xdx}{x+1}=$$
$$=e-1-\int_0^{e-1}\frac{x+1-1}{x+1}dx=e-1-\int_0^{e-1}dx+\int_0^{e-1}\frac{dx}{x+1}=$$
$$=e-1-(e-1)+\int_0^{e-1}\frac{d(x+1)}{x+1}=\ln(x+1)\Big|_0^{e-1}=\ln e-\ln 1=1.$$

Рассмотрим интеграл
\begin{equation}\label{eq:1}
\int_4^9\frac{2\sqrt{x}}{\sqrt{x}+1}dx
\end{equation}
Сделаем замену: $\sqrt{x}+1=t$, тогда $dt=d(\sqrt{x}+1)=\frac{dx}{2\sqrt{x}}=\frac{dx}{2(t-1)}$, откуда $dx=2(t-1)dt$. Получаем, что интеграл~\eqref{eq:1} равен
 $$\int_3^4\frac{2(t-1)\cdot 2(t-1)dt}{t}=4\int_3^4\frac{t^2-2t+1}{t}dt=4\int_3^4\left(t-2+\frac{1}{t}\right)dt=$$
 $$=4\left(\frac{t^2}{2}-2t+\ln|t|\right)\Big|_3^4=4\left(8-8+\ln4-\frac{9}{2}+6-\ln3\right)=6+4\ln\frac43.$$
 
Решим занесением под дифференциал:
$$\int_1^e\frac{\ln^3x}{x}dx=\int_1^e \ln^3xd(\ln x)=\frac{\ln^4x}{4}\Big|_1^e=\frac14-0=\frac14.$$
 
Рассмотрим интеграл $\int_0^{\pi/4}x \cdot \arcsin 2x dx$. Данный интеграл записан некорректно: $\arcsin 2x$ не существует в окрестности точки $\pi/4$ ($2x=\pi/2>1$). Построим график этой функции на рис.~\ref{graph1}:

\begin{figure}
\centering
\includegraphics[width=7cm]{xArcSin2x.jpg}
\caption{График функции $\int_0^{x}x \cdot \arcsin 2x dx$.}
\label{graph1}
\end{figure}

Для подготовки данного файла была использована книга~\cite{lvov}, а также книги~\cite{kudr, demid}

\begin{thebibliography}{9}

\bibitem{lvov}  Львовский С.М., Набор и верстка в пакете \LaTeX / 3-е издание, исправленное и дополненное. --- Москва : Московский центр непрерывного математического образования. --- 2003. --- С. 448.

\bibitem{kudr} Кудрявцев Н.Л., Лекции по математическому анализу. Часть I: Учебное пособие. --- М. :  ООО ``Сам полиграфист'',  2021. --- 256 с.

\bibitem{demid} Демидович Б.П., Сборник задач и упражнений по математическому анализу: Учебное пособие. --- 18-е изд., испр. --- М. : Изд-во Моск. ун-та, ЧеРо, 1997. --- 624 с.

\end{thebibliography}
 
\end{document} 